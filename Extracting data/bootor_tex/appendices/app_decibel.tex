The decibel (dB) expresses a ratio between two powers $\calP_1$ and $\calP_2$:
\begin{equation}
T_\dB = 10 \log_{10} \frac{\calP_1}{\calP_2}.
\label{eq:decibelDefinition}
\end{equation}
Hence, dB is a relative, not an absolute quantity. If $T_\dB=0$, then the powers are equal. If $T_\dB$ is positive, $\calP_1$ is larger than $\calP_2$ and vice-versa if $T_\dB<0$.

There are other definitions\footnote{See, e.\,g., \akurl{http://www.jimprice.com/prosound/db.htm}{BMdb}.} related to dB such as dBW, which is the relative power in dB of a signal with respect to 1~watt (W). An important distinction is that, while dB should always be a ``relative'' measure indicating a ratio of powers, dBW is an ``absolute'' measure that indicates one specific power value. For example, assuming that $\calP$ is power in watts, its conversion to $\hat \calP$ in dBW is given by
\[
	 \hat \calP = 10 \log_{10} \calP\textrm{~~~~~~~($\calP$ in W and $\hat \calP$ in dBW)}.
\]
For example, 100~W corresponds to 20~dBW.

Similarly, dBm is defined with respect to 1~milliwatt (mW). Assuming that $\calP$ is power in watts,
\[
	 \hat \calP = 10 \log_{10} \frac{\calP}{10^{-3}}\textrm{~~~~~($\calP$ in W and $\hat \calP$ in dBm)}.
\]
For example, if a receiver detects a signal level of $-13$~dBm, it means this signal has a power that is 13~dB smaller than a milliwatt. %Another example is that most mobile phones currently transmit signals at 27~dBm and the base station typically irradiates signals of power around 65~dBm.
For example, on a 3G cell phone using five ``bars'' to indicate signal strength, the 1-bar may correspond to $-113$~dBm and 5-bars to $-100$~dBm.

%AK I think it is confusing:
%Alternatively, in case $P$ is already in mW, one can use
%\[
%	 \hat P = 10 \log_{10} P\textrm{~~~~~~~(P in mW and $\hat P$ in dBm)}.
%\]

\ignore{
Para dar ideia:
Apos exaustiva pesquisa propria e com auxilio do CDG (CDMA Development Group), a
Alcatel-Lucent, apesar de nao ser fabricante de terminais, identificou que o limite proposto de 40
dBm seria excessivo, pois os terminais hoje disponiveis no mercado nao atingem este valor de
potencia, sendo que os mais potentes atingiriam algo em torno de 27 dBm.
Vale ressaltar que, mesmo considerando-se o valor proposto de 40 dBm para este limite, o limite
proposto no Art 5o deste Capitulo III, de 65 dBm para potencia efetivamente irradiada (e.i.r.p) de
uma estacao radio-base, seria
}

%Different than dB, powers in dBm, dBW, and other similar quantities become absolute values, not relative values anymore because there is a reference (absolute value). 

Another definition derived from dB is dBc, which is related to the presence of a carrier, such as in radio communications. In this case, the reference is the power of the strongest carrier. Therefore, typically, the dBc\index{dBc} value of a signal component is negative.

A more complicated definition is dBm0, used for example in audio and telephony. 
It means the level compared to a milliwatt after
the value is adjusted to make a reference (``correct'') value be 0 dBm.
\ignore{
%A good explanation by Floyd Davidson can be found at \url{http://yarchive.net/phone/decibels.html}, which is repeated here for convenience:
\begin{verbatim}
Let me try drawing a simple circuit to demonstrate.  Lets say
this is a one-way only circuit (so I don't have to put both
transmit and receive levels on it).  The part shown comes out of
a channel bank, goes to a pad, is then transported over a cable
pair to some distant point.  The levels shown are for a test
tone.
%
  +-----+      +-----+               +-----+      +-----+
  |     |      |     |     cable     |     |      |     |
  |Chan |      | 2 dB|  (6 dB loss)  |15 dB|      |     |
  |Bank |----->| PAD |->-----//----->| PAD |----->|Modem|
  |(Rcv)|  +7  |     | +5         -1 |     | -16  |     |
  |     | dBm  |     |dBm        dBm |     | dBm  |     |
  +-----+      +-----+               +-----+      +-----+
%
At each point, the level shown is the correct level for a test
tone, and is also the TLP.  Hence, the output of the channel
bank is called a +7 TLP, and the input to the modem is called a
-16 TLP.
%
If we measured this circuit, and at each point the levels were
exactly what is listed, then at each point the level would be
"0 dBm0".
%
But lets consider a more likely set of measurements!  The
channel bank output is actually a little low, it reads +6 dBm
(or -1 dBm0), the first pad isn't really 2 dB, it is 2.3 dB, so
the signal hitting the cable is 3.7 dBm (-1.3 dBm0), the cable
is wet today and therefore has 7.5 dB of loss instead of 6 dB so
the level out of the cable is -3.8 (-2.8 dBm0), and the pad
before the modem is actually a 14 dB pad instead of 15 dB, so
the level going into the modem is -17.8 (-1.8 dBm0).
%
(One way to demonstrate the usefulness of dBm0 is fabricate an
example like the above!  I guarantee that it was easiest to
decide how far off the particular loss was and then first
calculate how that would change the dBm0 value before figuring
the dBm value.  Figuring the dBm value before the dBm0 value is
just a lot harder.)
%
To show the usefulness of using dBm0 values, here is a table:
%
   Test Point     dBm0
   ChanBk Out     -1
   Cable Head     -1.3
   Cable End      -2.8
   Modem Input    -1.8
%
Each value shows how far from the correct value the level actually
is.  Here is a table of raw (called "uncorrected" readings),
%
   Test Point     dBm0    TLP
   ChanBk Out     +6      +7.0
   Cable Head     +3.7    +5.0
   Cable End      -3.8    -1.0
   Modem Input    -17.8   -16.0
%
Clearly another column is needed to indicate what the TLP is!
Otherwise that table is worthless.
%
Another practical example would be a leased line modem circuit
that goes across the country, with testboard access at 10
different locations.  If the level end-to-end is not correct,
the problem must be traced by checking with each testboard.  If
each testboard reports a dBm level, then they also must report
what the TLP if the location asking is to know if that level is
to high, too low, or just right.  By reporting a dBm0 value,
which is "corrected" for whatever the TLP level is, that just
eliminates confusion, because the value given is the amount by
which the signal is too high or too low.
\end{verbatim}
}
%There are many other units , such as dBrn0.
%, that will not be discussed here because are less common.

One unfortunate fact is that sometimes dB is erroneously interpreted as the absolute value of a signal power. For example, in sound engineering, the dB \emph{sound pressure level} or dB~SPL is widely used and it represents the ratio between the measured sound pressure level and the reference point. However,  the ``SPL'' suffix and the reference power value
are often omitted and dB may be incorrectly understood as an absolute power value in these cases.

In digital signal processing, most of the time the correct unit is unknown and, consequently,
there is no interest on specifying the reference value. If the signals are assumed to be
in volts and obtained from a resistance of 1~ohm, in many cases the correct unit would be dBW instead of dB. For example, in spectral analysis, it is common to convert a power spectral density $S(f)$
in W/Hz to $10 \log_{10} S(f)$, which should be interpreted as dBW/Hz. 
%However, in some cases the unit dB/Hz is used instead.
For example, the \ci{periodogram.m} function in {\matlab} shows graphs in dB/Hz 
but the informed unit could be dBW/Hz.

Sometimes dB is used to express voltage ratios, not power ratios. For example, assuming purely
resistive impedance $R$, the power associated to a voltage $V$ is $\calP = V^2/R$.
In this case \equl{decibelDefinition} can be written as
\begin{equation}
T_\dB = 10 \log_{10} \left( \frac{V_1^2/R}{V_2^2/R} \right)= 20 \log_{10} \left( \frac{V_1}{V_2} \right).
\label{eq:decibelAsVoltageRatios}
\end{equation}
Both voltage values in \equl{decibelAsVoltageRatios} should be measured over the same impedance. As an example of a possible mistake, consider an amplifier that requires 1~V from a
1,000~Ohms source to output 40~V over a 10~Ohms speaker. In this case, it is not strictly correct to use \equl{decibelAsVoltageRatios} and state that this amplifier has a ``gain'' of 
$20 \log_{10} (40/1) \approx 32$~dB. In fact, the power ratio in this case is $10 \log_{10} ((40^2/10)/(1/1000)) \approx 52$~dB.

There are also definitions of absolute values for voltage ratios such as dBmV, which for the cable industry represents decibels relative to 1~millivolt across 75~Ohms (the impedance of a coaxial cable). Hence, 0~dBmV corresponds to $10 \log_{10} ((10^{-3})^2/75) \approx -78.75$~dBW or -48.75~dBm.

Some other examples of dB usage:
\begin{itemize}
	\item To convert from dBW to dBm, simply add 30.
	\item If a signal is transmitted with power 5 dBm and goes through a channel that attenuates it by 3 dB, it arrives at the receiver with 2 dBm of power.
	\item A sinusoid with amplitude $A=10$ volts has average power $\calP=A^2/2 = 50$ W, which corresponds to $\approx 16.99$~dBW.
	\item A DC signal with amplitude $A=10$ volts, has average power 100 watts or, equivalently, 20 dBW.
	\item Assume one wants to design a (polar) signal that assumes only two amplitude values $-A$ and $A$ with power -9 dBm. The first step is to convert the power -9 dBm to $10^{-9/10}=0.126$ mW. Assuming $A$ is in volts, this signal has power $A^2$ watts and the amplitude should be $A = \sqrt{0.126 \times 10^{-3}} = 0.011$~V.
\end{itemize}

%AK-IMPROVE: more details about decibel and its related quantities, tal com o V/sqrt(Hz) para um dado valor em ohm
\ignore{
TO FINISH LATER:
%
	Uma outra interpretacao importante para o uso do dB eh quando se lida com funcoes de transferencia e respostas em frequencia. e comum a transformada de Laplace ser usada para expressar a funcao de transferencia H(s) = Y(s) / X(s), ou a de Fourier ser usada para expressar a resposta em frequencia do sistema H(f) = Y(f) / X(f) ou H(w) = Y(w) / X(w), em Hertz ou radianos por segundo, respectivamente. Como H(s) e H(f) (ou H(w)) sao funcoes complexas, usa-se comumente a forma polar de um numero complexo, expressando-se modulo e fase. Para o engenheiro, interessa saber, dado uma entrada X (normalmente uma funcao de numeros complexos), qual a saida Y. Na pratica, ha duas maneiras de expressar o modulo de H: a linear e a logaritmica (em dB). Considerando a resposta em frequencia (o mesmo vale para a funcao de transferencia), um filtro passa-baixa ideal com ganho 1 e frequencia de corte de 1.000 Hz pode ter o modulo do sinal de saida calculado fazendo-se a multiplicacao por 1 ou 0, dependendo da frequencia da componente do sinal. Outra maneira seria expressar H em dB. Neste caso o filtro teria ganho de 0 dB na faixa de 0 a 1.000 Hz. O sinal de entrada deveria ser expresso entao em dBm ou dBw, por exemplo, e o sinal de saida e encontrado fazendo-se a soma (ao inves do produto no caso linear) da entrada com a resposta em frequencia.
%	
	Uma "deducao" para isso seria: dado que se esta trabalhando com o modulo, tem-se $|Ylinear| = |Hlinear| |Xlinear|$. Supondo-se que a representacao linear e dada e deseja-se encontrar a representacao logaritmica em dB. Como o dB e uma razao entre duas potencias, tem-se: $|Hlogaritmica| = 10 \log_{10} (|Ylinear|^2 / |Xlinear|^2)  = 20 \log_{10} |Hlinear|$. Na pratica, a utilidade do dB e que se pode usar somas e subtracoes (ao inves de multiplicacoes e divisoes necessarias na representacao linear), dai e importante raciocinar como se pode representar sinais em dBm, por exemplo, e encontrar a saida em dBm. Isto pode ser entendido com outra "demonstracao" simples: pode-se elevar ao quadrado e tirar o logaritmo dos dois lados da expressao $|Ylinear| = |Hlinear| |Xlinear|$. Dado que se tem $10 log_{10} |Ylinear|^2 = 20 \log_{10} |Hlinear| + 10 \log_{10} |Xlinear|^2$, observa-se que isto corresponde a $YdBw = |Hlogaritmica| + XdBw$. Neste caso, X e Y seriam dados em dBw. Desejando-se trabalhar em dBm, pode-se pensar que bastaria dividir ambos lados de $|Ylinear| = |Hlinear| |Xlinear|$ por $10^{-3}$, apos elevar ao quadrado e seguir o mesmo raciocinio para encontrar $YdBm = |Hlogaritmica| + XdBm$.
}

\section{Insertion loss and insertion frequency response}
\label{sec:insertionLoss}

A system is often composed by the interconnection of several blocks and it is of interest to
know how a signal is attenuated (or amplified!) as it passes through the blocks.
The \emph{insertion loss}\index{Insertion loss} (IL) is widely used to characterize the loss in power, for example,
of a transmission line or a circuit. This section discusses the IL of a generic \emph{two-port network} (or \emph{quadripole}), which is any four-terminal network called
here \emph{device under test} (DUT).
Material about IL in specific applications can be found elsewhere. 
For example, in~\cite{T1417} 
(more specifically, ``C.3.1.7 Relationship of transfer function and \emph{insertion loss}'')
there is an interesting discussion of IL in the DSL systems context.

IL estimation is based on a smart strategy: measure the system output 
with and without the DUT, with the IL being the attenuation 
imposed by the DUT.

\begin{figure}
\centering
\includegraphics[width=\figwidthSmall,keepaspectratio]{FiguresNonScript/insertionLoss}
\caption{Basic setup for measuring the insertion loss of a device under test (DUT).\label{fig:insertionLoss}}
\end{figure}

\figl{insertionLoss} depicts a setup for measuring the insertion loss of a DUT.
The behavior without the DUT is obtained by removing it (i.\,e.,
directly connecting the pairs of terminals 1 and 2, and measuring the 
power over the load, which in this case will be denoted as $\calP_L^{\textrm{no}}$.
Then, the behavior with the DUT is obtained by inserting it 
and again measuring the power over the load, which will be denoted as $\calP_L^{\textrm{yes}}$.
The insertion loss in dB is defined as
\begin{equation}
\textrm{IL} \defeq \frac {\calP_L^{\textrm{no}}} {\calP_L^{\textrm{yes}}}
\label{eq:insertionLossPower}
\end{equation}
and is often given in dB:
\begin{equation}
\textrm{IL}_{\textrm{dB}} \defeq 10 \log_{10} \left( \frac {\calP_L^{\textrm{no}}} {\calP_L^{\textrm{yes}}} \right).
\label{eq:insertionLossPowerdB}
\end{equation}

For example, a passive optical device that splits the input into two output signals could have an IL of 4~dB per output, consisting of the ideal 3~dB IL for splitting the power in halves plus an \emph{excess loss} of 1~dB per port.

If the system behavior varies with frequency, $\textrm{IL}(f)$ can be measured and reported over the frequency range of interest.
Besides, in case the DUT amplifies its input, one
can use the insertion gain $\textrm{IG}(f) = 1/\textrm{IL}(f)$ instead.

Another alternative definition of IL uses
the voltage $V_L$ over the load. Repeating the procedure
of removing and then inserting the DUT, the measured values of
$V_L$ are denoted as $V_L^{\textrm{no}}$ and $V_L^{\textrm{yes}}$,
respectively. In this case and assuming frequency-dependence, IL can be written as
\begin{equation}
\textrm{IL}(f)= 10 \log_{10} \frac {|V_L^{\textrm{no}}(f)|^2} {|V_L^{\textrm{yes}}(f)|^2}  =20 \log_{10} \frac {|V_L^{\textrm{no}}(f)|} {|V_L^{\textrm{yes}}(f)|}.
\label{eq:insertionLossVoltage}
\end{equation}

%\begin{equation}
%H_{\textrm{IL}}(f) = \frac {V_L^{\textrm{no}}(f)} {V_L^{\textrm{yes}}(f)},
%\label{eq:insertionLoss}
%\end{equation}
%which is often expressed in dB as
%\begin{equation}
%20 \log_{10} |H_{\textrm{IL}}(f)|.
%\label{eq:insertionLossIndB}
%\end{equation}
%
%In case the DUT amplifies its input, one
%can use the insertion gain $H_{\textrm{IG}}(f) = 1/H_{\textrm{IL}}(f)$
%instead.

The ``insertion'' trick (measuring with and without the DUT) can be used to obtain parameters other than the IL. For example, in some situations it is inconvenient to
deal with (or difficult to measure) 
the frequency response $H(f)$ directly because it depends on 
the source and load impedances. In these cases, it may be advantageous
to measure the \emph{insertion frequency response}\index{Insertion frequency response} 
\begin{equation}
H_{\textrm{I}}(f) = \frac {V_L^{\textrm{yes}}(f)} {V_L^{\textrm{no}}(f)}
\label{eq:insertionFreqResponse}
\end{equation}
and convert it to $H(f)$ when convenient and the source $Z_s$ and load $Z_l$ impedances
are known. Note that \equl{insertionFreqResponse} does not discard the phases of the measured voltages as in \equl{insertionLossVoltage}, and can be obtained for example with a network analyzer equipment. 
%The procedure to obtain $H(f)$ from $H_{\textrm{I}}(f)$ is the following.

Having $H_{\textrm{I}}(f)$, the frequency response $H(f)$ relating the output voltage $V_L^{\textrm{yes}}(f)$ (with the DUT) to the input source voltage $V_S(f)$ can be obtained as
\begin{equation}
H(f) \defeq \frac {V_L^{\textrm{yes}}(f)} {V_s(f)}  = \frac {V_L^{\textrm{no}}(f)} {V_s(f)} \frac {V_L^{\textrm{yes}}(f)} {V_L^{\textrm{no}}(f)} = \frac {Z_L(f)} {Z_S(f) +
Z_L(f)}H_{\textrm{I}}(f).	
	\label{eq:frequencyResponseFromHI}
\end{equation}
When $Z_L(f)$ and $Z_S(f)$ can be considered constant over the frequency of interest, \equl{frequencyResponseFromHI} is conveniently expressed as
\begin{equation}
H(f) = \frac {Z_L} {Z_S +
Z_L} H_{\textrm{I}}(f).
	\label{eq:frequencyResponseFromHI2}
\end{equation}

$H_{\textrm{I}}(f)$ itself is of interest in some applications. For example, when estimating the channel capacity of a DSL copper loop, $H_{\textrm{I}}(f)$ is adopted instead of $H(f)$~\cite{T1417}.

The ``insertion'' procedure is used in many distinct scenarios but people tend to keep together the terms ``insertion'' and ``loss'', instead of using only the former. 
%there are several definitions of IL in the literature. 
For example, some authors (e.\,g. \cite{Ciofficn} and in \figl{saw_filter2}) call insertion \emph{loss} to $\textrm{IG}(f)$ and report an attenuation as a negative value in dB.
Others call the insertion frequency response of \equl{frequencyResponseFromHI2} as ``insertion loss'', as in the FTW software, or ``loop insertion gain transfer function'' as in~\cite{T1417}.