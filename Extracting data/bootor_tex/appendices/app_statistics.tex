\subsection{Confidence intervals}
%
The code below indicates how to calculate confidence intervals. The application is the misclassification error of a classifier.
%
\begin{lstlisting}
%Input data
confidence = 80 % in percent, e.g., 95 %
n = 4  %total number of examples in the test set
m = 2  %number of errors (or matches)
%Auxiliary terms
p = m/n
q = 1-p
alpha = (1 - confidence/100) / 2
zc = norminv(1-alpha) %this could be found via a table
% Simple method
d = zc*sqrt(p*q/n)
inferiorLimit = p - d
superiorLimit = p + d
% More advanced method based on a 2nd order equation 
tempb = zc * sqrt( ((p*q)/n) + ((zc^2)/(4*n^2)) );
tempa = p + (zc^2/(2*n));
den = 1 + ( (zc^2) / n);
superior = (tempa + tempb) / den
inferior = (tempa - tempb) / den
\end{lstlisting}
