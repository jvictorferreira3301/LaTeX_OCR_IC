%This package must be friendly to all users at LaPS
%If a command is agressive to the template you are using, take it out, but comment in the SVN log, so that other users will know what was excluded.
%For example, I (AK) moved all commands "\usepackage" to the file usepackages.tex.

\newcommand\arrowedbox[1] {{\rightarrow\boxed{#1}\rightarrow}} %draw box with arrow

%Command "define" from http://www.latex-community.org/forum/viewtopic.php?f=46&t=15406
\newcommand\defeq{\stackrel{\text{def}}{=}}
\newcommand\defeqsmall{\stackrel{\mathsmaller{\mathsf{def}}}{=}}

%use labels starting with eq: and fig: for equations and figures
\newcommand\blol[1] {{Block~(\ref{eq:#1})}}
\newcommand\equl[1] {{Eq.~(\ref{eq:#1})}}
\newcommand\figl[1] {{Figure~\ref{fig:#1}}}
\newcommand\tabl[1] {{Table~\ref{tab:#1}}}
%use labels starting with ak_ for listings
\newcommand\codl[1] {{Listing~\ref{code:#1}}}
\newcommand\exal[1] {{Example~\ref{ex:#1}}}


%Expected value
\newcommand\ev{\mathbb{E}}

\renewcommand{\Re}{{\mathbb R}}
\newcommand\ebt{{\varepsilon_b}}
\newcommand\eba{{\overline\varepsilon_b}}
%\newcommand\ekt{{\varepsilon_k}}
\newcommand\ekt{\varepsilon_{\scriptscriptstyle K}}
\newcommand\eka{\overline\varepsilon_{\scriptscriptstyle K}}
\newcommand\deck{{F}}
\newcommand\decb{{f}}
\newcommand\bH{{\bf H}}
\newcommand\bM{{\bf M}}
\newcommand\bw{{\bf w}}
\newcommand\bt{{\bf t}}
\renewcommand{\Re}{{\mathbb R}}
%\def \svmlight {{$\text{SVM}^{Light}$}}
\def \abM    {{$\alpha$~vs.~$\beta$}}
\def \svmlight {{$\text{SVM}^{Light}$}}
%\newcommand\eka{\overline\varepsilon_{\scriptscriptstyle K}}
%\def\x{{\mathbf x}}
%\def\L{{\cal L}}
%\def \abM    {{$\alpha$~vs~$\beta$}}
%\def \svmlight {{$\text{SVM}^{Light}$}}
%\newcommand\Num {\textrm{Num}}
%\newcommand\Den {\textrm{Den}}
%\newcommand\tji {{\theta_{ji}}}
%\newcommand\tki {{\theta_{ki}}}
%\newcommand\aij {{a_{ij}}}
%\newcommand\Nji {{N_{ji}}}
%\def \kernelxxi {\calK(\bx,\bx_i)}
%\newcommand\sign {\textrm{sign}}
%\newcommand\bonea{\overline{B_1}}
\newcommand\ci[1] {\textsf{#1}}
\newcommand\co[1] {\textsf{#1}}

\newcommand\myattention[1] {\textcolor{red}{\LARGE #1}}

\def \rvX     {{\bf X}}
\def \rvY     {{\bf Y}}
\newcommand\Rd      {{R^{d}}}
\newcommand\Rg      {{R^{g}}}
\newcommand\Thetad      {{\Theta^{d}}}
\newcommand\Thetag      {{\Theta^{g}}}

\newcommand\bSigma{\mbox{\boldmath$\Sigma$}}
\newcommand\bmu{\mbox{\boldmath$\mu$}}
\newcommand\bpi{\mbox{\boldmath$\pi$}}
\newcommand\bnu{\mbox{\boldmath$\nu$}}

%this one is beautiful for partial derivative:
\def \parder#1#2{{ \frac {\partial #1} {\partial #2}}}
%\def \tp#1#2#3{{\theta_{{#1 #2}}^{#3}}}
\newcommand\tp[3]{{\theta_{{#1 #2}}^{#3}}}

\ifdefined\bm
\else
\newcommand\bm[1]{{\mbox{\boldmath $#1$}}}
\fi

\newcommand\Line[0]{%
  \rule{0cm}{0cm}\\\hrule\rule{0cm}{0cm}%
}

%Alon Orlitsky
% Theorem-like environments

\def \qed     {\Box}  % here because other styles don't use box.

\ifdefined\Theorem
\else
\newtheorem{Theorem}{Theorem}
\fi

\newtheorem{Theorem*}{Theorem}

\newtheorem{Application}{Application}
\newtheorem{Application*}{Application}
\newtheorem{Claim}{Claim}
\newtheorem{Claim*}{Claim}

\ifdefined\Corollary
\else
\newtheorem{Corollary}[Theorem]{Corollary}
\fi

\newtheorem{CounterExample*}{$\overline{\hbox{\bf Example}}$}

\ifdefined\Example
\else
\newtheorem{Example}{Example}
\fi

\newtheorem{Example*}{Example}
\newtheorem{Exercise}{Exercise}
\newtheorem{Intuition*}{Intuition}
\newtheorem{Joke*}{Joke}

\ifdefined\Lemma
\else
\newtheorem{Lemma}[Theorem]{Lemma}
\fi


\newtheorem{Lemma*}{Lemma}
\newtheorem{Model}{Model}
\newtheorem{Model*}{Model}
\newtheorem{Open problem}{Open problem}
\newtheorem{Proposition}{Proposition}
\newtheorem{Property}{Property}
\newtheorem{Question}{Question}
\newtheorem{Question*}{Question}
\newtheorem{Remark*}{Remark}
\newtheorem{Result}{Result}

% Begin / End Theorems

\def \bthm#1{\begin{#1}\upshape\quad}
\def \ethm#1{\rqed\end{#1}}
\def \ethmp#1{\end{#1}}     % No box, when ends in displayed equation

%AK This has been redefined in dsl_definitions.tex in .\dslbook. If you need them
%copy and paste into your tex. Do not uncomment them here.
%\def \bApplication {\bthm{Application}}
%\def \eApplication {\ethm{Application}}
\def \bClaim     {\bthm{Claim}}
\def \eClaim     {\ethm{Claim}}
\def \eClaimp    {\ethmp{Claim}}
\def \bCorollary {\bthm{Corollary}}
\def \eCorollary {\ethm{Corollary}}
\def \eCorollaryp{\ethmp{Corollary}}
\def \bCounter   {\bthm{CounterExample*}}
\def \eCounter   {\ethm{CounterExample*}}
\def \eCounterp  {\ethmp{CounterExample}}
%AK This has been redefined in dsl_definitions.tex in .\dslbook. If you need them
%copy and paste into your tex. Do not uncomment them here.
%\def \bExample   {\bthm{Example}}
%\def \eExample   {\ethm{Example}}
\def \eExamplep  {\ethmp{Example}}
\def \bExercise  {\bthm{Exercise}}
\def \eExercise  {\ethm{Exercise}}
\def \eExercisep {\ethmp{Exercise}}
\def \bIntuition {\bthm{Intuition}}
\def \eIntuition {\ethmp{Intuition}}
\def \bJoke      {\bthm{Joke}}
\def \eJoke      {\ethm{Joke}}
\def \bLemma     {\bthm{Lemma}}
\def \eLemma     {\ethm{Lemma}}
\def \eLemmap    {\ethmp{Lemma}}
\def \bModel     {\bthm{Model}}
\def \eModel     {\ethm{Model}}
\def \bOpen      {\bthm{Open problem}}
\def \eOpen      {\ethm{Open problem}}
\def \eOpenp     {\ethmp{Open problem}}
\def \bProperty  {\bthm{Property}}
\def \eProperty  {\ethm{Property}}
\def \bQuestion  {\bthm{Question}}
\def \eQuestion  {\ethm{Question}}
\def \bRemark    {\bthm{Remark}}
\def \eRemark    {\ethm{Remark}}
%defined in dsl_definitions
%\def \bResult    {\bthm{Result}}
%\def \eResult    {\ethm{Result}}
\def \bTheorem   {\bthm{Theorem}}
\def \eTheorem   {\ethm{Theorem}}
\def \eTheoremp  {\ethmp{Theorem}}

\def \bSubexa    {\begin{subexa}}
\def \eSubexa    {\ethm{subexa}}

% Environments

\ifdefined\Problem
\else
\newenvironment{Problem}{\textbf{Problem}\\ \begin{enumerate}}{\end{enumerate}}
\fi


\newenvironment{Problems}{\textbf{Problems}\\ \begin{enumerate}}{\end{enumerate}}

%\newenvironment{Problems}{\begin{trivlist}\item[]{\textbf{Problems}}{\end{trivlist}}}

% Headers

\def \skpbld#1{\par\noindent\textbf{#1}\quad}
\def \skpblds#1{\skpbld{#1}\par\noindent}

\def \Answer   {\skpbld{Answer}}
\def \Answers  {\skpblds{Answers}}
\def \Basis    {\skpbld{Basis}}
\def \Check    {\skpbld{Check}}
\def \Intuition{\skpbld{Intuition}}
\def \Method   {\skpbld{Method}}
\def \Methods  {\skpblds{Methods}}
\def \Outline  {\skpbld{Outline}}
\def \Proof    {\skpbld{Proof}}
\def \Proofs   {\skpblds{Proofs}}
%\def \Problem  {\skpbld{Problem}}
%\def \Problems {\skpblds{Problems}}
\def \Proline  {\skpbld{Proof Outline}}
\def \Remark   {\skpbld{Remark}}
\def \Remarks  {\skpblds{Remarks}}
\def \Solution {\skpbld{Solution}}
\def \Solutions{\skpblds{Solutions}}
\def \Verify   {\skpbld{Verify}}
\def \Step     {\skpbld{Step}}

% Ignores

\newcommand\ignore[1]{{}}

%\newcommand\problem[1]{#1}
%\newcommand\problem[1]{}

%\newcommand\solution[1]{\mbox{}\\ \medskip\noindent{\bf Solution\medskip}#1}
%\newcommand\solution[1]{{\bf Solution}\quad #1}
%\newcommand\solution[1]{#1}
%\newcommand\solution[1]{}

%\newcommand\source[1]{(Taken from~\cite{#1})}
%\newcommand\source[1]{}

\newcommand\takenfrom[1]{(Taken from~\cite{#1})}

% Formating

\newcommand\joke[1]{\footnote{#1}}
\newcommand\trivia[1]{\footnote{#1}}

% Handouts

\newcommand\hohead[3]{
\hfill\begin{minipage}{3.5cm}
\bf
#1\\ % class # (CSE20) (perhaps + term)
#2\\ % term (Fall 2K+1) or name (Alon Orlitsky)
%Handout #3\\
\end{minipage}
}

\newcommand\solhead[3]{ % CSE20 -- Discrete Math; Fall 2K+1; 1
\begin{center}
  \Large #1\\[1ex]
  \normalsize #2\\[3ex]
  Solutions for Homework Assignment \##3\\
\end{center}
}

% Blackboard fonts

\newcommand\EE{\mathbb{E}}
\newcommand\CC{\mathbb{C}}
\newcommand\NN{\mathbb{N}} % \dN
\newcommand\QQ{\mathbb{Q}} % \dQ
\newcommand\RR{\mathbb{R}} % \dR
\newcommand\ZZ{\mathbb{Z}} % \dZ

% Numbers

\newcommand\complex{\CC}
\newcommand\integers{\ZZ}
\newcommand\naturals{\NN}
\newcommand\rationals{\QQ}
\newcommand\reals{\RR}
\newcommand\RRp{\reals^+}
\newcommand\integersp{\ZZ^+}
\newcommand\integerss[1]{\ZZ_{\ge{#1}}}

% boldface
\def \bzeroM {{\bf 0}}
\def \bff     {{\bf f}}

\def \bA {{\textbf A}}
\def \bB {{\textbf B}}
\def \bC {{\textbf C}}
\def \bD {{\textbf D}}
\def \bE {{\textbf E}}
\def \bF {{\textbf F}}
\def \bG {{\textbf G}}
\def \bH {{\textbf H}}
\def \bI {{\textbf I}}
\def \bJ {{\textbf J}}
\def \bK {{\textbf K}}
\def \bL {{\textbf L}}
\def \bM {{\textbf M}}
\def \bN {{\textbf N}}
\def \bO {{\textbf O}}
\def \bP {{\textbf P}}
\def \bQ {{\textbf Q}}
\def \bR {{\textbf R}}
\def \bS {{\textbf S}}
\def \bT {{\textbf T}}
\def \bU {{\textbf U}}
\def \bV {{\textbf V}}
\def \bW {{\textbf W}}
\def \bX {{\textbf X}}
\def \bY {{\textbf Y}}
\def \bZ {{\textbf Z}}

\def \ba {{\textbf a}}
\def \bb {{\textbf b}}
\def \bc {{\textbf c}}
\def \bd {{\textbf d}}
\def \be {{\textbf e}}
\def \bf {{\textbf f}}
\def \bg {{\textbf g}}
\def \bh {{\textbf h}}
\def \bi {{\textbf i}}
\def \bj {{\textbf j}}
\def \bk {{\textbf k}}
\def \bl {{\textbf l}}
\def \bm {{\textbf m}}
\def \bn {{\textbf n}}
\def \bo {{\textbf o}}
\def \bp {{\textbf p}}
\def \bq {{\textbf q}}
\def \br {{\textbf r}}
\def \bs {{\textbf s}}
\def \bt {{\textbf t}}
\def \bu {{\textbf u}}
\def \bv {{\textbf v}}
\def \bw {{\textbf w}}
\def \bx {{\textbf x}}
\def \by {{\textbf y}}
\def \bz {{\textbf z}}

\def \bmx {{\mathbf x}}

% caligraphics

\def \calA {{\cal A}}
\def \calB {{\cal B}}
\def \calC {{\cal C}}
\def \calD {{\cal D}}
\def \calE {{\cal E}}
\def \calF {{\cal F}}
\def \calG {{\cal G}}
\def \calH {{\cal H}}
\def \calI {{\cal I}}
\def \calJ {{\cal J}}
\def \calK {{\cal K}}
\def \calL {{\cal L}}
\def \calM {{\cal M}}
\def \calN {{\cal N}}
\def \calO {{\cal O}}
\def \calP {{\cal P}}
\def \calQ {{\cal Q}}
\def \calR {{\cal R}}
\def \calS {{\cal S}}
\def \calT {{\cal T}}
\def \calU {{\cal U}}
\def \calV {{\cal V}}
\def \calW {{\cal W}}
\def \calX {{\cal X}}
\def \calY {{\cal Y}}
\def \calZ {{\cal Z}}

\def \cala {{\cal a}}
\def \calb {{\cal b}}
\def \calc {{\cal c}}
\def \cald {{\cal d}}
\def \cale {{\cal e}}
\def \calf {{\cal f}}
\def \calg {{\cal g}}
\def \calh {{\cal h}}
\def \cali {{\cal i}}
\def \calj {{\cal j}}
\def \calk {{\cal k}}
\def \call {{\cal l}}
\def \calm {{\cal m}}
\def \caln {{\cal n}}
\def \calo {{\cal o}}
\def \calp {{\cal p}}
\def \calq {{\cal q}}
\def \calr {{\cal r}}
\def \cals {{\cal s}}
\def \calt {{\cal t}}
\def \calu {{\cal u}}
\def \calv {{\cal v}}
\def \calw {{\cal w}}
\def \calx {{\cal x}}
\def \caly {{\cal y}}
\def \calz {{\cal z}}

% vectors

%\def \veca  {{\overline a}}
%\def \vecb  {{\overline b}}
%\def \vecX  {{\overline X}}
%\def \vecY  {{\overline Y}}
%\def \vecx  {{\overline x}}
%\def \vecy  {{\overline y}}
%\def \vec#1{{\overline{#1}}}

% Abbreviations

\newcommand\eg{\textit{e.g.,}\xspace}
\newcommand\etc{etc.\@\xspace}
\newcommand\ie{\textit{i.e.,}\xspace}  % note that overridden in spanish

% notes

%\definecolor{light}{gray}{.75}

\def \mynote#1{{}}

% marginal notes

\def \marcha{{\marginpar{CHNGD}}}
\def \marfix{{\#\marginpar{FIX \#}}}
\def \marnew{{\marginpar{NEW}}}
\def \marok{{\#\marginpar{\# OK?}}}

% qed's

\def \eqed    {\eqno{\qed}}
\def \rqed    {\hbox{}~\hfill~$\qed$}

% sequences

\def \upto  {{,}\ldots{,}}
\def \zn    {0\upto n}
\def \znmo  {0\upto n-1}
\def \znpo  {0\upto n+1}
\def \ztnmo {0\upto 2^n-1}
\def \on    {1\upto n}
\def \onmo  {1\upto n-1}
\def \onpo  {1\upto n+1}

% sets

\def \setpmo   {\{\pm 1\}}
\def \setmpo   {\{-1{,}1\}}
\def \setzo    {\{0{,}1\}}
\def \setzn    {\{\zn\}}
\def \setznmo  {\{\znmo\}}
\def \setztnmo {\{\ztnmo\}}
\def \seton    {\{\on\}}
\def \setonmo  {\{\onmo\}}
\def \set#1#2{{\{{#1}\upto{#2}\}}}
\def \setzon   {\setzo^n}
\def \setzos   {\setzo^*}

\def \sets#1{{\{#1\}}}
\def \Sets#1{{\left\{#1\right\}}}

\def \inseg#1{{[#1]}}

% functions

\def \ord    {\#}

\def \suml   {\sum\limits}
\def \prodl  {\prod\limits}

% Set operations

%\def \union  {\cup}
\def \union  {\bigcup}
\def \unionl {\union\limits}
%\def \Union {\Bigcup} % want this
\def \Union  {\bigcup}
\def \Unionl {\Union\limits}

\def \inter  {\cap}
%\def \inter  {\bigcap}
\def \interl {\inter\limits}
\def \Inter {\Bigcap}
\def \Interl {\Inter\limits}

% Floors and Ceilings

\def \ceil#1{{\lceil{#1}\rceil}}
\def \Ceil#1{{\left\lceil{#1}\right\rceil}}
\def \floor#1{{\lfloor{#1}\rfloor}}
\def \Floor#1{{\left\lfloor{#1}\right\rfloor}}

% Parentheses, brackets

\def \paren#1{{({#1})}}
\def \Paren#1{{\left({#1}\right)}}
\def \brack#1{{[{#1}]}}
\def \Brack#1{{\left[{#1}\right]}}

%\def \frac#1#2{{{#1}\over{#2}}}

\def \binomial#1#2{{{#1}\choose{#2}}}

\def \gcd#1#2{{{\rm gcd}\paren{{#1},{#2}}}}
\def \Gcd#1#2{{{\rm gcd}\Paren{{#1},{#2}}}}

\def \lcm#1#2{{{\rm lcm}\paren{{#1},{#2}}}}
\def \Lcm#1#2{{{\rm lcm}\Paren{{#1},{#2}}}}

\newcommand\ed{\stackrel{\mathrm{def}}{=}}
%\def \ed     {\,{\buildrel \rm def \over =}\,}
\def \gap    {\ \AQUIVOUEU{\raisebox{-.6ex}{$\stackrel{\textstyle>}{\sim}$}}\ }

% values

\def \half    {{\frac12}}
\def \quarter {{\frac14}}
\def \oo#1{{\frac1{#1}}}

% notation

\def \pr     {{p}}
\def \Pr     {{\hbox{Pr}}}
\def \iff    {{\it iff }}
\def \th     {{\rm th }}

% ignore

\def\ignore#1{}

% Logic

%\newcommand\ra{\rightarrow}
\def \contra {{\leftrightarrows}}
\newcommand\ol[1]{{\overline{#1}}}
%\def \ob {\overline}

\def \ve {{\lor}} % needed?
\def \eq {{\equiv}} % needed?

% spaces

\newcommand\spcin{\hspace{1.0in}}
\newcommand\spchin{\hspace{.5in}}

% Default text appears in regular print in both book and class versions.
% There are two types of text that need to be highlighted in class:
% clson - not mentioned in book version (eg jokes)
% clsbk - regular text in book (the parts that need be said)

%For book:
%\newcommand\clson[1]{}
%\newcommand\clsbk[1]{#1}
%For class:
\newcommand\clson[1]{\colorbox{light}{{#1}}}
\newcommand\clsbk[1]{\colorbox{light}{{#1}}}

%\newcommand\bi{\begin{aopl}}
%\newcommand\ei{\end{aopl}}

%\newcommand\bi{\begin{itemize}}
%\newcommand\ei{\end{itemize}}
%\newcommand\bq{\begin{quote}}
%\newcommand\eq{\end{quote}}

\newenvironment{aopl}
  {\begin{list}{}{\setlength{\itemsep}{4pt plus 2pt minus 2pt}}}
  {\end{list}}

% operators

\def\orpro{\mathop{\mathchoice
   {\vee\kern-.49em\raise.7ex\AQUIVOUEU{$\cdot$}\kern.4em}
   {\vee\kern-.45em\raise.63ex\AQUIVOUEU{$\cdot$}\kern.2em}
   {\vee\kern-.4em\raise.3ex\AQUIVOUEU{$\cdot$}\kern.1em}
   {\vee\kern-.35em\raise2.2ex\AQUIVOUEU{$\cdot$}\kern.1em}}\limits}

\def\andpro{\mathop{\mathchoice
 {\wedge\kern-.46em\lower.69ex\AQUIVOUEU{$\cdot$}\kern.3em}
 {\wedge\kern-.46em\lower.58ex\AQUIVOUEU{$\cdot$}\kern.25em}
 {\wedge\kern-.38em\lower.5ex\AQUIVOUEU{$\cdot$}\kern.1em}
 {\wedge\kern-.3em\lower.5ex\AQUIVOUEU{$\cdot$}\kern.1em}}\limits}

\def\inter{\mathop{\mathchoice
   {\cap}
   {\cap}
   {\cap}
   {\cap}}}
\def\interl {\inter\limits}

\def\carpro{\mathop{\mathchoice
   {\times}
   {\times}
   {\times}
   {\times}}\limits}

% for old documents

%\newcommand\twlrm{\fontsize{12}{14pt}\normalfont\rmfamily}
%\newcommand\tenrm{\fontsize{10}{12pt}\normalfont\rmfamily}

% picture macros

\newcommand\grid[2]{ % grid{width}{height}
  \multiput(0,0)(1,0){#1}{\line(0,1){#2}}
  \put(#1,0){\line(0,1){#2}}
  \multiput(0,0)(0,1){#2}{\line(1,0){#1}}
  \put(0,#2){\line(1,0){#1}}
}

%AK: to write in red the section titles
%https://tex.stackexchange.com/questions/10555/hyperref-warning-token-not-allowed-in-a-pdf-string
%There's the aptly, if verbosely, named macro \texorpdfstring, which takes two arguments and uses the first for (La)TeX and the second for pdf

%problem below is that sections etc are not recognized
%\newcommand\advancedsection[1] {{\section{\texorpdfstring{\textcolor{red}{Advanced: }#1}{Advanced: #1}}}}
%\newcommand\advancedsubsection[1] {{\subsection{\texorpdfstring{\textcolor{red}{Advanced: }#1}{Advanced: #1}}}}
%\newcommand\advancedsubsubsection[1] {{\subsubsection{\texorpdfstring{\textcolor{red}{Advanced: }#1}{Advanced: #1}}}}

%\newcommand\akadvanced {{\texorpdfstring{\textcolor{red}{Advanced:}}{Advanced:}}}
%\newcommand\akadvanced{{\texorpdfstring{\textcolor{red}{Advanced:}}{Advanced:}}}\xspace
\newcommand\akadvanced{{\texorpdfstring{\textcolor{red}{Advanced:}}{Advanced:}}{}}

\def \nfft    {{N_{\textrm{fft}}}}

\def \Nperiod {{N_0}} %period of discrete-time signals, avoid to use N
\def \Tperiod {{T_0}} %period of continuous-time signals, avoid to use T

%\def \digif {{\mathbb{F}}} %discrete-time linear frequency %AK: I stopped using it


%from http://tex.stackexchange.com/questions/42726/align-but-show-one-equation-number-at-the-end
\newcommand\numberthis{\addtocounter{equation}{1}\tag{\theequation}}

\def \nuv {{\boldsymbol{\nu}}} %boldface \nu. assumes amsmath package as in  http://tex.stackexchange.com/questions/595/how-can-i-get-bold-math-symbols

\def \cconv {{\circledast}} %\varoast \oast %circular convolution

\newcommand\figwidth{.7\textwidth} %defaut fig width
\newcommand\figwidthSmall{.5\textwidth} %smaller fig width
\newcommand\figwidthLarge{.9\textwidth} %larger fig width

\newcommand{\ip}}[2]{{ \langle {#1} , {#2} \rangle }} %inner product. Usage example: $\ip{\varphi_j(t)}{\varphi_k(t)} = 1$

%This command allows to use \href when compiling with pdflatex or disable it when compiling for a DVI with latex
%I use it to create a list of URLs, which I will later identify on a web site via urls_akbook.tex
\newcommand{\akurl}}[2]{{\ifpdf{[\href{{#1}}{{url#2}}]}\else{\IfPackageLoadedTF{tex4ht}{[\href{{#1}}{{url#2}}]}{[url#2]}}\fi}} %

\def \sign {{\textrm{sign}}} %sign function

\newcommand\sgn{\mathop{\mathrm{sgn}}}

\def \aktilde {{$\sim$}} 
%\def \aktilde {{\textrm{\textasciitilde}}} %this is too high, there is a nice \midtilde in wsuipa package but did not want to include the package
\def \muV {{$\mu$V}}  %micro Volts (1e-6)
\def \mus {{$\mu$s}}  %micro seconds (1e-6)

%distinguish digital and analog frequencies
%\def \dw {{\omega}}  %in radians
%\def \aw {{\Omega}}  %in radians/sec
\def \aw {{\omega}}  %in radians
\def \dw {{\Omega}}  %in radians/sec

%AK problem: when I have math in a \section I have problems with the PDF compilation, which tries to create a hyperlink. I had to put \ifpdf. I tried to create a macro below but did not succeded.
\ignore{
\newcommand\oldaktexorpdfstring[2]}{{%
\ifpdf
   \ifHy@pdfstring
     \expandafter\@secondoftwo
   \else
     \expandafter\@firstoftwo
   \fi
\else
	\expandafter{#1}
\fi
}}
%
\newcommand\notworkingaktexorpdfstring[2]}{{
\ifpdf
	{alo pai \expandafter{\texorpdfstring{#1}{#2}}}
\else
	{oi mae {#1}}
\fi
}}
}

\newcommand\bit{\begin{itemize}}
\newcommand\eit{\end{itemize}}

%\newlist{exercises}{enumerate}{2}
%\setlist[exercises]{label=\thechapter.\arabic*.,font=\small\color{red},widest=99,leftmargin=0pt,itemindent=2.4em,align=left,itemsep=.5ex}

%\newlist{application}{enumerate}{2}
%\setlist[application]{label=Application \thechapter.\arabic*.,font=\small\color{red}
%,widest=99,leftmargin=0pt,itemindent=2.4em,align=left,itemsep=.5ex}
%\def \bApplication {\begin{application} {\item}}
%\def \eApplication {\end{application}}

%\def \bApplication {\subsection{Application}}
%\def \eApplication {\end{Application}}

%AK created an environment! With a counter!
\newcounter{appcounter}
%\newenvironment{name}[num]{before}{after}
%\newenvironment{application}{font=\small\color{red} Application \thechapter.\arabic*. }{}
\newenvironment{application}{
\bigskip\noindent
\refstepcounter{appcounter}
{\textbf{\textcolor{red}{Application \theappcounter.}} }}{\rqed} %use box in end
\def \bApplication {\begin{application}}
\def \eApplication {\end{application}}
\numberwithin{appcounter}{chapter} %uses amsmath according to http://tex.stackexchange.com/questions/28333/continuous-v-per-chapter-section-numbering-of-figures-tables-and-other-docume

%AK created an environment! With a counter!
\newcounter{exacounter}
%\newenvironment{name}[num]{before}{after}
%\newenvironment{application}{font=\small\color{red} Application \thechapter.\arabic*. }{}
\newenvironment{example}{
\bigskip\noindent
\refstepcounter{exacounter}
{{\textcolor{red}{Example \theexacounter.}} }}{\rqed \bigskip} %use box in end
\def \bExample {\begin{example}}
\def \eExample {\end{example}}
\numberwithin{exacounter}{chapter} %uses amsmath according to 

%AK created another environment! With a counter!
\newcounter{rescounter}
%\newenvironment{name}[num]{before}{after}
%\newenvironment{application}{font=\small\color{red} Application \thechapter.\arabic*. }{}
\newenvironment{akresult}{
\bigskip\noindent
\refstepcounter{rescounter}
{{\textcolor{green}{Result \therescounter.}} }}{\rqed \bigskip} %use box in end
\def \bResult {\begin{akresult}}
\def \eResult {\end{akresult}}
\numberwithin{rescounter}{chapter} %uses amsmath according to 



%\def \bApplication {\bthm{Application}}
%\def \eApplication {\ethm{Application}}

\def \bModel     {\bthm{Model}}
\def \eModel     {\ethm{Model}}

%#######################
%Shortcuts
%#######################
\def \xrms {{x_{\textrm{rms}}}} %RMS value of x
\def \BW {{\textrm{BW}}} %bandwidth
\def \MSE {{\textrm{MSE}}} %mean-squared error
\def \SNR {{\textrm{SNR}}} %SNR

\def \dB {{\textrm{dB}}} %decibel
\def \sp {{\textrm{~}}}

%\newcommand\arrowedbox[1] {{\rightarrow\fbox{#1}\rightarrow}}
\def \setM {{\mathscr M}}

%\real and \imag for real and imaginary parts
\newcommand\real[1]{{\mathcal Real \{#1\}}}
\newcommand\imag[1]{{\mathcal Imag \{#1\}}}


%Similarly, discussions on comp.text.tex have revealed that there are a variety of
%ways to indicate the mathematical notion of "is defined as". Common candidates include:
%\triangleq), \equiv), \coloneqq) and:
%\def \define {{{\stackrel{\text{\tiny def}}{=}}}}  %def gets too close of left-side
%\def \define {{\triangleq}}

\def \cm {{Common mistake}}

\def \tmax {{{\textrm{max}}}}
\def \tmin {{{\textrm{min}}}}

%random signals
\def \rsx {{\mathsf x}}
\def \rsy {{\mathsf y}}
\def \rsz {{\mathsf z}}

%random variables
\def \rvn {{\mathsf N}}
\def \rvr {{\mathsf R}}
\def \rvk {{\mathsf K}}
\def \rve {{\mathsf E}}
\def \rvx {{\mathsf X}}
\def \rvy {{\mathsf Y}}
\def \rvz {{\mathsf Z}}
%\def \rvx {{\mathbb X}}
%\def \rvy {{\mathbb Y}}
%\def \rvz {{\mathbb Z}}

\def \conv {{\ast}}  %convolution
%\def \conv {{\convolution}}  %package mathabx convolution

%moved to definitions.tex
%use labels starting with eq: and fig: for equations and figures
%\newcommand\blol[1] {{Block~(\ref{eq:#1})}}
%\renewcommand{\equl}[1] {{Eq.~(\ref{eq:#1})}}
%\renewcommand{\figl}[1] {{Figure~\ref{fig:#1}}}
%\newcommand\tabl[1] {{Table~\ref{tab:#1}}}
%use labels starting with ak_ for listings
%\newcommand\codl[1] {{Listing~\ref{code:#1}}}
%\newcommand\exal[1] {{Example~\ref{ex:#1}}}

%expected value:
\renewcommand{\ev}{\mathbb{E}}
%\def \expval {\textbf{E}}
\def \expval {\ev}

%#######################
%Need to distinguish
%#######################
\def \ts {{T_s}}%sampling period
\def \fs {{F_s}}%sampling frequency
\def \tsym {{T_{\textrm{sym}}}}%symbol period
\def \rsym {{R_{\textrm{sym}}}} %symbol rate
\def \tdmt {{T_{\textrm{dmt}}}}%symbol period
\def \rdmt {{R_{\textrm{dmt}}}} %symbol rate
\def \xa {{$x(t)$}} %analog
\def \xd {{$x_q[n]$}}  %digital
\def \xs {{$x_s(t)$}} %sampled
\def \xsarea#1{{{x_s^{\textrm{area}}({{#1}})}}} %area of sampled signal

%#######################
%Code listing
%#######################
%Making easier to include the listing of source code:
%assume name starts with ak_
%Example:
%directory, name
%\includecode{MatlabOctaveFunctions}{dctmtx}
%is equivalent to
%\lstinputlisting[caption={MatlabOctaveFunctions/ak\_dctmtx.m},label=code:dctmtx]{./Code/MatlabOctaveFunctions/ak_dctmtx.m}
\def \includecode#1#2 {{\lstinputlisting[caption={#1/ak\_#2.m},label=code:#2]{./Code/#1/ak_#2.m}}}

%need to take care of the cases where an underscore _ exists or the cases where ak is not the prefix
%directory, caption, name
\def \includecodelong#1#2#3 {{\lstinputlisting[caption={#1/#2.m},label=code:#3]{./Code/#1/#3.m}}}
%Examples:
%\includecodelong{MatlabOctaveFunctions}{qpsk\_simple}{qpsk_simple}
%\includecodelong{MatlabOctaveFunctions}{ak\_gram\_schmidt}{ak_gram_schmidt}
%In case the macros do not work, use:
%\lstinputlisting[language=Python,caption=MatlabBookFigures/figs\_transforms\_dctimagecoding,label=code:dctimagecoding]{./Code/MatlabBookFigures/figs_transforms_dctimagecoding.m}

%now add hyperlink to Python version
\newcommand{\akpythoncode}}[2]{{\ifpdf{[\href{{#1}}{{#2}}]}\else{\IfPackageLoadedTF{tex4ht}{[\href{{#1}}{{#2}}]}{[#1]}}\fi}} %

\def \includecodepython#1#2#3 {{\lstinputlisting[caption={#1/#2.m. \akpythoncode{https://github.com/aldebaro/dsp-telecom-book-code/blob/master/PythonCodeSnippets/#2.py}{Python version}},label=code:#3]{./Code/#1/#3.m} }}


%#######################
%Others
%#######################

%three-dimensional array "channel matrix"
\newcommand\cmH{{ \overline {\textbf H}}}

\def \matlab {{{Matlab/Octave}}}

%total and max
%\def \ptot {{p^{\textrm{tot}}}}
\def \ptot {{{\calP}^{\textrm{tot}}}}

\def \pmax {{P^{\textrm{max}}}}
\def \rtarget {{r^{\textrm{target}}}}
%allow to provide subscript
\newcommand\ptotn[1] {{p_{#1}^{\textrm{tot}}}}
\newcommand\pmaxn[1] {{P_{#1}^{\textrm{max}}}}
\newcommand\rtargetn[1] {{r_{#1}^{\textrm{tar}}}}

\def \grandmatrix {{\overline{\mathbf G}}}

\newcommand\nk[1] {{#1_n^k}}

\def \calR {{\cal R}}
\def \rrb {\overline{\cal R}|}
\def \txtand {{\textrm{~and~}}}
\def \snr {{\textrm{SNR}}} %signal to noise ration
\def \snrdb {{\textrm{SNR}_{\textrm{dB}}}}

%\def \ae {\varepsilon_x}
\def \ae {{E_c}}
\def \ab {\overline{b}}
\def \no { {\cal N}_0 }  %should not it be \N0 with zero
\def \dfreq {e^{j \omega}}

%margin:
%\setlength{\parindent}{0em} \setlength{\oddsidemargin}{0cm}
%\setlength{\evensidemargin}{0cm} \setlength{\textwidth}{16.5cm}

\newcommand\IR{\rm I\hspace{-0.06cm}R}
\newcommand\IN{\rm I\hspace{-0.06cm}N}

%\newcommand\ra{\;\rightarrow\;}
\newcommand\lra{\;\Leftrightarrow\;}
\newcommand\fa{\;\forall\;}
\newcommand\ex{\;\exists\;}

\newcommand\sinc{{\textrm{sinc}}}
\newcommand\rect{{\textrm{rect}}}
%\def \rect {{\textrm{rect}}} %rect function

\newcommand\angledeg[1]{\angle{{#1}}^{\circ}}
\newcommand\degree[1]{{{#1}}^{\circ}}